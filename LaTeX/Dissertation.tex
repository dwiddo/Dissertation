\documentclass[11pt]{article}
\usepackage[left=1.2in,right=1.2in,top=1.2in,bottom=1.2in]{geometry}
\usepackage{lipsum}
\usepackage[hidelinks]{hyperref}
\urlstyle{same}
\usepackage{multicol}
\usepackage{footmisc}
  
\usepackage{mathtools}
\usepackage{amsmath}
\usepackage{amsfonts}
\usepackage{amssymb}
\usepackage{gensymb}

\usepackage[utf8]{inputenc}

% For nicely formatted header (choose what the left, right and center headers and footers say) (\thepage gives the page number, \leftmark gives section title and \rightmark gives the subsection title)
\usepackage{fancyhdr}
\pagestyle{fancy}
\lhead{}
\chead{}
\rhead{Daniel Widdowson, 201459067}
\lfoot{}
\cfoot{\thepage}
\rfoot{}
\renewcommand{\headrulewidth}{0.4pt}
\renewcommand{\footrulewidth}{0.4pt}
\renewcommand{\thefootnote}{[\arabic{footnote}]}
\title{Dissertation}
\author{Daniel Widdowson}

\begin{document}

\maketitle

\section{Summary}

The aim of this project is to investigate a specific challenge in computational geometry, the motivation for which lies in crystallography. The project will be carried out with a general eye towards developing new methods for visualisation, analysis and prediction of crystal structures. 

\begin{thebibliography}{9}
\bibitem{Mosca,Kurlin} 
M. Mosca and V. Kurlin. \emph{Algorithmic challenges of computational geometry motivated by applications in crystallography}. 2020.
\end{thebibliography}

 
\end{document}